\documentclass[conference]{IEEEtran}
\IEEEoverridecommandlockouts
% The preceding line is only needed to identify funding in the first footnote. If that is unneeded, please comment it out.
\usepackage{cite}
\usepackage{amsmath,amssymb,amsfonts}
\usepackage{algorithmic}
\usepackage{graphicx}
\usepackage{textcomp}
\usepackage{xcolor}
\usepackage{balance}

\usepackage[nohyperlinks, printonlyused, nolist]{acronym}

\usepackage[utf8]{inputenc}
\usepackage[T1]{fontenc} % Trennen von Wörtern mit Umlauten
\usepackage{ngerman} % Damit z.B. "Literatur" statt "References" da steht


\def\BibTeX{{\rm B\kern-.05em{\sc i\kern-.025em b}\kern-.08em
    T\kern-.1667em\lower.7ex\hbox{E}\kern-.125emX}}
\begin{document}

\begin{acronym}
    \acro{apt}[APT]{Advanced Persistent Threat}
\end{acronym}


\title{Conference Paper Title
}

\author{\IEEEauthorblockN{Maria Mustermann}
    \IEEEauthorblockA{\textit{Fakultät für Informatik} \\
        \textit{Technische Hochschule Rosenheim}\\
        Rosenheim, Germany \\
        maria.mustermann@stud.th-rosenheim.de}
}

\maketitle

\begin{abstract}
    Als \acp{apt} werden Gruppierungen oder Personen bezeichnet, die über ein hohes Maß an Fachwissen und erhebliche Ressourcen verfügen, die es Ermöglichen mehrere Angriffsvektoren zu nutzen.
    Zusätzlich verfolgen \acp{apt} ihre Ziele wiederholt über einen längeren Zeitraum, passen sich den Bemühungen der Verteitiger an und sind entschlossen die Ziele zu erreichen.
    Zu diesen Zielen gehören u.\nobreakspace a. das Exfiltrieren von Informationen, kritische Aspekte einer Organisation zu stören und sich im System des Ziels zu verbreiten und festzusetzen \cite[S.~B-1]{NIST2011}.
    Die geschätzten jährlichen Kosten von Cyberkriminalität sollen laut Statista im Jahr 2023 auf 8,15 Billionen US-Dollar belaufen und bis 2028 um 69,94\% auf 13.82 Billionen US-Dollar steigen \cite{Statista2023}.
    Deshalb ist es für Organisationen und Firmen unerlässlich sich gegen \acp{apt} und anderen Akteuren zu schützen.

    Paper soll anhand der APT XX zeigen, wie diese Vorgehen, welche Techniken verwendet werden und welche Ziele APTs verfolgen.
    Im Schlussteil werden die wichtigsten Maßnahmen aufgezeigt oder eine raffinierte Maßnahme im Detail erklärt
\end{abstract}

\begin{IEEEkeywords}
    component, formatting, style, styling, insert
\end{IEEEkeywords}

\section{Introduction}

\section{Ease of Use}

\subsection{Maintaining the Integrity of the Specifications}

\section{Prepare Your Paper Before Styling}

\subsection{Abbreviations and Acronyms}

\subsection{Units}


\subsection{Equations}

\subsection{\LaTeX-Specific Advice}


\subsection{Some Common Mistakes}
\cite{Cole2013}
\cite{NIST2011}
\cite{Statista2023}
\subsection{Authors and Affiliations}


\subsection{Identify the Headings}

\subsection{Figures and Tables}
\paragraph{Positioning Figures and Tables}



\section*{Acknowledgment}

\section*{References}

\balance
\bibliographystyle{IEEEtran}
\bibliography{IEEEexample}



\vspace{12pt}
\color{red}
IEEE conference templates contain guidance text for composing and formatting conference papers. Please ensure that all template text is removed from your conference paper prior to submission to the conference. Failure to remove the template text from your paper may result in your paper not being published.

\end{document}
